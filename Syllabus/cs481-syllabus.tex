\documentclass[11pt]{article}

\usepackage[pdftex,bookmarks=false]{hyperref}
\usepackage{moreverb}
\usepackage{boxedminipage}
\usepackage{syllabus}

%%%%%%%%%%%%%%%%%%%%%%%%%%%%%%%%%%%%%%%%%%%%%%%%%%

\begin{document}

\SyllabusHeader{Spring 2020}{CS 481/Math 481 Seminar}
\SyllabusTimeRoomOfficeHours{10:00-10:50 Monday}{BHSN 358}
{Monday 1:30pm-2:30pm,  Tuesday 8:30am-9:30am, Wednesday 8:30am-9:00am,} { Thursday 8:30am-9:30am, 1:30pm-2:30pm, Friday 8:30am-9:00am and by appointment.}

%%%%%%%%%%%%%%%%%%%%%%%%%%%%%%%%%%%%%%%%%%%%%%%%%%

\SyllabusCatalogDescription{%
The study of topics not included in the usual curriculum;
presentations by students doing research; lectures by visiting
scientists and faculty. Open only to juniors and seniors or by special
permission.
}

%%%%%%%%%%%%%%%%%%%%%%%%%%%%%%%%%%%%%%%%%%%%%%%%%%

\SyllabusPrerequisitesCredits{None}{1}%

Text: None

%%%%%%%%%%%%%%%%%%%%%%%%%%%%%%%%%%%%%%%%%%%%%%%%%%

\SyllabusBeginGoals{}

\item
To develop independent learning and research skills in topics beyond the standard major curriculum. This meets the University learning goal of developing independent, lifelong learners.

\item
To provide opportunities for students to gain experience in planning and delivering oral presentations. This meets the University learning goal of communicating clearly.

\item
To improve technical writing skills. This meets the University learning goal of communicating clearly.

\item
To provide opportunities for students to attend academic conferences and/or present their research to regional and national conferences and submit their work for publication. This meets the University learning goals of developing independent, lifelong learners and becoming independent, lifelong learners.

\SyllabusEndGoals{}

%%%%%%%%%%%%%%%%%%%%%%%%%%%%%%%%%%%%%%%%%%%%%%%%%%

\SyllabusBeginOutcomes{}

\item
Students will explore topics beyond the standard curriculum via a conference, paper, or presentation. This will be assessed by the conference report, research paper, or oral presentation.

\item
Students will develop their communications skills. This will be assessed by the quality of the paper or presentation. 

\SyllabusEndOutcomes{}

%%%%%%%%%%%%%%%%%%%%%%%%%%%%%%%%%%%%%%%%%%%%%%%%%%

\SyllabusReading{varies, depending on project}

%%%%%%%%%%%%%%%%%%%%%%%%%%%%%%%%%%%%%%%%%%%%%%%%%%

\SyllabusAssignments{}

\SyllabusWebPage{}

\flushleft
\begin{tabular}[t]{|l|r|}
\hline
Resume & 10\%\\\hline
Paper/Presentation & 80\%\\\hline
Attendance/Participation & 10\%\\\hline
%Resume & 10\%\\\hline
\end{tabular}%
\hspace{0.1in}%
\begin{tabular}[t]{|l|r|}
\hline
Expected outside of class work & Hours\\\hline\hline
Paper/Presentation & 30\\\hline
%Resume & 3\\\hline
\end{tabular}

\newpage

\SyllabusGradingScale

%%%%%%%%%%%%%%%%%%%%%%%%%%%%%%%%%%%%%%%%%%%%%%%%%%
\flushleft
\begin{bfseries}Resume Workshop\end{bfseries}

%Eric Anderson from Career Services will be giving an interview skills workshop in class on Sep 10th.

Eric Anderson from Career Services will be giving a resume workshop in class on Jan 27th. You must submit a resume via email as an attachment (either .docx or .pdf) with the subject \begin{bfseries}resume\end{bfseries} to Eric (eanderson@capital.edu and Cc: lfeng@capital.edu) by Monday Jan 20th at 5PM) so that Eric can review them beforehand. Please see the guidelines for writing a resume at: \url{http://www.capital.edu/uploadedFiles/Content/Academics/Services_and_Programs/Career_Development/How\%20to\%20Write\%20a\%20Resume.pdf?n=9138}. \\

You can also get help by making an appointment with Peer Career Advisors through Handshake. Go to the Handshake login page: \url{https://capital.joinhandshake.com/login}. Enter your Capital email and follow the direction. If you have trouble, DO NOT CREATE A NEW LOGIN/PROFILE (instead, contact Career Development).\\

After logging in, you can make an appointment:
\begin{enumerate}
\item
Click on "Career Center" at the top of your computer screen (or click on "School" on the bottom of your mobile device screen), and select "Appointments"
\item
Click on "Schedule a New Appointment", and follow the prompts!
\end{enumerate}
On Jan 27th, you may bring a hardcopy or a laptop and Eric will have suggestions for improvements. Submitting it by Monday Jan 20th at 5PM and bringing it to class on Jan 27th will be 10\% of your seminar grade.

\flushleft
\begin{bfseries} Interview Skills Workshop\end{bfseries}

Eric Anderson will give an interview skills workshop in class on Feb 3rd.


\flushleft
\begin{bfseries} Recruiter Talk\end{bfseries}

A recruiter from TEKSystems will talk about job opportunities and requirements on Feb 10th.

\flushleft
\begin{bfseries} Technical Talk\end{bfseries}

Dr. Nicholas Van Horn from Psychology Department will give a technical talk on his research work on Feb 17th. 
\url{http://www.massmine.org}


%%%%%%%%%%%%%%%%%%%%%%%%%%%%%%%%%%%%%%%%%%%%%%%%%%

\flushleft
\begin{bfseries}Presentation/Paper/Conference Requirements\end{bfseries}

During your two or three semesters of Math/CS 481, you must give one presentation and write at least one research paper. If you are required to take three semesters of Math/CS481, you may attend a conference one semester. The intention is that students will start a research project and write a paper on their progress and then continue working on it and present their work during one of their last two semesters of Math/CS481. We also encourage students to present their work at Capital's Symposium on Undergraduate Research and conferences such as the spring MAA conference or the National Conference on Undergraduate Research.

\flushleft

Presentations must be 20-25 minutes in length (18-20 minutes if there are two presentations on the same day) and include the following: introduction/overview, outline of talk, details of research/development, summary/future work, and a bibliography. The bibliography must include at least five relevant sources and at least two of these must be books or journal articles (unless your topic advisor indicates otherwise). Presentation must have ``slides'' (typically created using PowerPoint) that are readable and contain appropriate content. A blue background with a white or yellow font is recommended for readability. Do not write out everything you are going to say. It is not necessary to have handouts, but if you have tables, figures, code fragments, etc. that you want handed out to the audience, please get them to me by the Wednesday before your presentation and I will have the division's secretary make copies at the department's expense. If your handouts are not ready by the Wednesday before your presentation, you are responsible for making your own copies.

\flushleft

While any quality presentation may include some material that is beyond the understanding of the general audience, you should make certain that the audience understands the basic idea of what you are presenting. In many cases it is appropriate to include slides showing some details, but not spend a significant amount of time presenting them. For example, if part of your work involves solving a differential equation, it is appropriate to show the equation and your
solution and give an overview of how you solved it, but do not go through the solution process in great detail. Similarly, if your work involves writing a computer program, you may show some of the code,
but do not explain what each line of code does.  Make your slides as readable as possible. Try to concentrate on the big picture and what you think your audience members should remember after the presentation is completed. The rubric for evaluating presentations will be provided to you on iLearn. You may earn an additional 5-10 percentage points on your presentation if the department agrees it is appropriate for presenting at Capital's scholarship symposium in the spring or at an external conference and you give the second presentation.

\flushleft

Papers must be 8-10 pages in length (single-spaced with a 11 or 12 pt font). If your paper requires figures or images, do not include these in the page totals. The bibliography counts for up to one page of text and the same guidelines for references mentioned earlier for presentations apply. The rubric for evaluating papers will be posted on iLearn. 
%Papers for the fall semester are due on December 2nd at the beginning of class.
%Papers for the spring semester are due on April 23rd at the beginning of class.

\flushleft

If you attend a conference, you must attend at least two hours of talks and provide a half-page summary of each talk with one relevant question you had regarding each talk. You must get approval for the specific conference from an advisor, meet with the advisor before the conference to get approval for the talks you will be attending, and once following the conference to discuss your summaries of the talks. 

\FL
\needspacelines{5}
\bold{Conference Lists}\\

\begin{tabular}{@{}ll}
%OCTM & Ohio Council of Teachers of Mathematics, Akron, OH, Oct 11 - 12, 2018 \\ 
%& \url{http://www.ohioctm.org/conferences/68th-annual-conference-akron-2018}\\
%EUSRC & Electronic Undergraduate Statistics Research Conference. Online, Nov 2, 2018\\
%& \url{https://www.causeweb.org/usproc/eusrc/2018/program/}\\
%WSDS & Women in Statistics and Data Science Conference, Cincinnati, OH Oct 18 - 20, 2018\\
%& \url{http://ww2.amstat.org/meetings/wsds/2018/}\\
MAA & Ohio MAA Conference,  April 3-4, Bowling Green State University - Bowling Green, OH\\
%& \url{https://www.ohio.edu/eastern/community/MAA-Conference.cfm}\\
%AMC & Annual Mathematics Conference, Miami University, Sep 21-22, 2018 \\
%& \url{http://miamioh.edu/cas/academics/departments/mathematics/about/events/annual-mathematics-conference/index.html}\\
%OCWiC19 & Ohio Celebration of Women in Computing, Huron, Feb 22 - 23, 2019 \\
%& \url{https://ocwic.hosting.acm.org/ocwic19/}\\
%GDEX &  The Midwest's Premier Gaming Expo, Columbus, OH, Dev Day Sep 28, Expo 29 - 30th, 2018\\
%& \url{https://www.thegdex.com}\\
\end{tabular}

Talk to your picked advisor for more details
%Conference summaries for the fall semester are due on December 2nd at the beginning of class. 
%Conference summaries for the spring semester are due on April 23rd at the beginning of class.

\flushleft

For each option, you must have with a faculty member from the department who agrees to advise you on your project. There is an ``advising sheet'' that you must fill out at each meeting and have the professor sign. The requirements for meetings for each option are:

\flushleft
Presentations:\\
\begin{tabular}{@{}|l|l|l|}
\hline
Deadline & Requirement & Penalty\\\hline
Fri 1/24 at noon & pick topic w/ approval & 5 points off\\\hline
6 weeks before presentation & update & 5 points off\\\hline
4 weeks before presentation & update & 5 points off\\\hline
1 week before presentation & rehearse & 10 points off\\\hline
Thursday before presentation at 5PM & abstract & 10 points off\\\hline
\end{tabular}

\flushleft
Papers:\\
\begin{tabular}{@{}|l|l|l|}
\hline
Deadline & Requirement & Penalty\\\hline
%Fri 9/7 at noon & pick topic w/ approval & 5 points off\\\hline
%9/24 & outline and bibliography & 5 points off\\\hline
%10/15  & 1/3 of rough draft & 5 points off\\\hline
%11/12 & complete rough draft & 10 points off\\\hline
%11/26 & paper due & 20 points off\\\hline

Fri 1/24 at noon & pick topic w/ approval & 5 points off\\\hline
2/10 & outline and bibliography & 5 points off\\\hline
3/16  & 1/3 of rough draft & 5 points off\\\hline
4/6 & complete rough draft & 10 points off\\\hline
4/20 & paper due & 20 points off\\\hline
\end{tabular}

\flushleft
Conference:\\
\begin{tabular}{@{}|l|l|l|}
\hline
Deadline & Requirement & Penalty\\\hline
Fri 1/24 at noon & pick conference w/ approval & 5 points off\\\hline
one week before conference & choose talks w/ approval & 10 points off\\\hline
within 2 weeks after conference & rough draft of write-ups & 10 points off\\\hline
4/21 & paper due & 20 points off\\\hline
\end{tabular}

\SyllabusPolicies{}

%\FL
%All students in their final academic year at Capital must take the
%subject exam for their major(s). The exam will be given during the
%final exam period (Monday Apr 30th
%10:15-12:30) and will affect your fall seminar grade (or next semester of seminar if the score is not available immediately). You may move the test to next semester if you are taking a course in your major that is covered on the test. The subject exam is a national exam and is scored using a percentile system. The following table lists the scores and the amount that will be added/subtracted from your seminar grade.
%
%\begin{center}
%\begin{tabular}{@{}|l|l|}
%\hline
%Exam Score & Seminar Grade\\
%\hline
%70-99th Percentile & +6\%\\
%60-69th Percentile & +4\%\\
%50-59th Percentile & +2\%\\
%30-49th Percentile & 0\%\\
%20-29th Percentile & -2\%\\
%10-19th Percentile & -4\%\\
%Below 10th Percentile & -10\%\\
%\hline
%\end{tabular}
%\end{center}
%
%Graduating seniors must participate in the exit interview at the end of the spring semester. Failure to
%participate in the exit interviews will result in 20 percentage points
%being deducted from your spring seminar grade.

\SyllabusSeminarAttendance

\SyllabusPhone{}

\SyllabusSeminarIntegrity{}

\SyllabusAcademicSuccess{}

\SyllabusDisability

\SyllabusTitleIX{}

%%%%%%%%%%%%%%%%%%%%%%%%%%%%%%%%%%%%%%%%%%%%%%%%%%

\FL
\needspacelines{7}
\bold{Advisor Topic Lists}\\

\begin{tabular}{@{}ll}
Dr. Federico & Math modeling in ecology, infectious diseases/epidemiology, and other applications \\ 
& using differential equations, matrices, or individual-based models\\
Dr. Feng & see PDF on iLearn\\
Dr. Johnson & \url{https://docs.google.com/document/d/1w-DwMi_0Pki4JH_vcka3DyrgoCLcL5K6tCtJosPw_ro/edit?usp=sharing}\\
Dr. Reed & see PDF on iLearn\\
Dr. Shields & computational modeling of physical systems \\
Dr. Stadler & \url{https://www.dropbox.com/s/tlplcbpygqoamxz/SeminarTopicsStadler.pdf?dl=0}\\

\end{tabular}


%%%%%%%%%%%%%%%%%%%%%%%%%%%%%%%%%%%%%%%%%%%%%%%%%%

%\SyllabusWebPage{CS481}

%\SyllabusSchedule{October 19}
\SyllabusSchedule{March 13}

%%%%%%%%%%%%%%%%%%%% CUT HERE %%%%%%%%%%%%%%%%%%%%

%%%%%%%%%%%%%%%%%%%%%%%%%%%%%%%%%%%%%%%%%%%%%%%%%%

\begin{tabularx}{\linewidth}{|r|X|}

%%%%%%%%%%%%%%%%%%%%%%%%%%%%%%%%%%%%%%%%%%%%%%%%%%

\hline
 & Monday
%%%%%%%%%%%%%%%%%%%%%%%%%%%%%%%%%%%%%%%%%%%%%%%%%%
\\\hline\hline
%%%%%%%%%%%%%%%%%%%%%%%%%%%%%%%%%%%%%%%%%%%%%%%%%%

% Week 1
1
&

1/6 - Course overview
\\\hline

%%%%%%%%%%%%%%%%%%%%%%%%%%%%%%%%%%%%%%%%%%%%%%%%%%

% Week 2
2
&

1/13 - Meet with Faculty to pick a topic 
\\\hline

%%%%%%%%%%%%%%%%%%%%%%%%%%%%%%%%%%%%%%%%%%%%%%%%%%

% Week 3
3
&

1/20 - Martin Luther King's Day (No class) (Resume Due today and Signup sheet Due Friday 1/24)
\\\hline

%%%%%%%%%%%%%%%%%%%%%%%%%%%%%%%%%%%%%%%%%%%%%%%%%%

% Week 4
4
&

1/27 - Resume Workshop By Eric Anderson
\\\hline

%%%%%%%%%%%%%%%%%%%%%%%%%%%%%%%%%%%%%%%%%%%%%%%%%%

% Week 5
5
&

2/3 - Interview Skill Workshop By Eric Anderson
\\\hline

%%%%%%%%%%%%%%%%%%%%%%%%%%%%%%%%%%%%%%%%%%%%%%%%%%

% Week 6
6
&

2/10 - Recruiter Talk By TEKsystems
\\\hline

%%%%%%%%%%%%%%%%%%%%%%%%%%%%%%%%%%%%%%%%%%%%%%%%%%

% Week 7
7
&

2/17 - Technical Presentation By Dr. Nicholas Van Horn 
\\\hline

%%%%%%%%%%%%%%%%%%%%%%%%%%%%%%%%%%%%%%%%%%%%%%%%%%

% Week 8
8
&

2/24 -  Spring Break (No class)
\\\hline

%%%%%%%%%%%%%%%%%%%%%%%%%%%%%%%%%%%%%%%%%%%%%%%%%%

% Week 9
9
&

3/2 - Update with your picked advisor (No class)
\\\hline

%%%%%%%%%%%%%%%%%%%%%%%%%%%%%%%%%%%%%%%%%%%%%%%%%%

% Week 10
10
&

3/9 - Presentation: Eryn Bernard, Tom Green
\\\hline

%%%%%%%%%%%%%%%%%%%%%%%%%%%%%%%%%%%%%%%%%%%%%%%%%%

% Week 11
11
&

3/16 - Class was canceled
\\\hline

%%%%%%%%%%%%%%%%%%%%%%%%%%%%%%%%%%%%%%%%%%%%%%%%%%

% Week 12
12
&

3/23 - Presentation: Elli Wachtman, Nate Boone
\\\hline

%%%%%%%%%%%%%%%%%%%%%%%%%%%%%%%%%%%%%%%%%%%%%%%%%%

% Week 13
13
&

3/30 - Presentation: Andrew Harmon, Samantha Shultz
\\\hline

%%%%%%%%%%%%%%%%%%%%%%%%%%%%%%%%%%%%%%%%%%%%%%%%%%

% Week 14
14
&

4/6 - Presentation: Grant Miller, Jarrett Williams
\\\hline

%%%%%%%%%%%%%%%%%%%%%%%%%%%%%%%%%%%%%%%%%%%%%%%%%%

% Week 15
15
&

4/13 - Presentation: Ryan Chaffins, Makayla Jones
\\\hline

%%%%%%%%%%%%%%%%%%%%%%%%%%%%%%%%%%%%%%%%%%%%%%%%%%

%Week 16
16
&

4/20 - Presentation: Ryan Ripley
\\\hline

%%%%%%%%%%%%%%%%%%%%%%%%%%%%%%%%%%%%%%%%%%%%%%%%%%
\end{tabularx}



%%%%%%%%%%%%%%%%%%%% CUT HERE %%%%%%%%%%%%%%%%%%%%

\end{document}
